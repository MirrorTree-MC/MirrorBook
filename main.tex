\documentclass[lang=cn,10pt]{elegantbook}

\title{MirrorBook:基于 \LaTeX{} 的新手指引}
\subtitle{棱镜之书 - 序章}

\author{Various Artists}
\institute{Mirror\LaTeX{} Program}
\date{January 6, 2025}
\version{0.1}
% \bioinfo{自定义}{信息}

\extrainfo{不要以为抹消过去,重新来过,即可发生什么改变。—— 比企谷八幡}

\setcounter{tocdepth}{3}

\logo{logo.png}
\cover{lyc_church.jpeg}

% 本文档命令
\usepackage{array}
\newcommand{\ccr}[1]{\makecell{{\color{#1}\rule{1cm}{1cm}}}}

% 修改标题页的橙色带
% \definecolor{customcolor}{RGB}{32,178,170}
% \colorlet{coverlinecolor}{customcolor}

\begin{document}

\maketitle
\frontmatter

\tableofcontents

\mainmatter

\chapter{Elegant\LaTeX{} 系列模板介绍}

Elegant\LaTeX{} 项目组致力于打造一系列美观、优雅、简便的模板方便用户使用。目前由 \href{https://github.com/ElegantLaTeX/ElegantNote}{ElegantNote},\href{https://github.com/ElegantLaTeX/ElegantBook}{ElegantBook},\href{https://github.com/ElegantLaTeX/ElegantPaper}{ElegantPaper} 组成,分别用于排版笔记,书籍和工作论文。强烈推荐使用最新正式版本!本文将介绍本模板的一些设置内容以及基本使用方法。如果您有其他问题,建议或者意见,欢迎在 GitHub 上给我们提交 \href{https://github.com/ElegantLaTeX/ElegantBook/issues}{issues} 或者邮件联系我们。

我们的联系方式如下,建议加入用户 QQ 群提问,这样能更快获得准确的反馈,加群时请备注 \LaTeX{} 或者 Elegant\LaTeX{} 相关内容。
\begin{itemize}
  \item 官网:\href{https://elegantlatex.org/}{https://elegantlatex.org/}(暂时歇业)
  \item GitHub 地址:\href{https://github.com/ElegantLaTeX/}{https://github.com/ElegantLaTeX/}
  \item Gitee 地址:\href{https://gitee.com/ElegantLaTeX}{https://gitee.com/ElegantLaTeX}
  \item CTAN 地址:\href{https://ctan.org/pkg/elegantbook}{https://ctan.org/pkg/elegantbook}
  \item 下载地址:\href{https://github.com/ElegantLaTeX/ElegantBook/releases}{正式发行版},\href{https://github.com/ElegantLaTeX/ElegantBook/archive/master.zip}{最新版}
  \item 微博:Elegant\LaTeX{}(密码有点忘了)
  \item 微信公众号:Elegant\LaTeX{}(不定期更新)
  \item 用户 QQ 群:692108391(建议加群)
  \item 邮件:\email{elegantlatex2e@gmail.com}
\end{itemize}

\section{模板安装与更新}

你可以通过免安装的方式使用本模板,包括在线使用和本地(文件夹内)使用两种方式,也可以通过 \TeX{} 发行版安装使用。

\subsection{在线使用模板}

我们把三套模板全部上传到 \href{https://www.overleaf.com/}{Overleaf} 上了,网络便利的用户可以直接通过 Overleaf 在线使用我们的模板。使用 Overleaf 的好处是无需安装 \TeX{} Live,可以随时随地访问自己的文件。查找模板,请在 Overleaf 模板库里面搜索 \lstinline{elegantlatex} 即可,你也可以直接访问\href{https://www.overleaf.com/latex/templates?addsearch=elegantlatex}{搜索结果}。选择适当的模板之后,将其 \lstinline{Open as Template},即可把模板存到自己账户下,然后可以自由编辑以及与别人一起协作。更多关于 Overleaf 的介绍和使用,请参考 Overleaf 的\href{https://www.overleaf.com/learn}{官方文档}。

\subsection{本地免安装使用}

\textbf{免安装}使用方法如下:从 GitHub 或者 CTAN 下载最新版,严格意义上只需要类文件 \lstinline{elegantbook.cls}。然后将模板文件放在你的工作目录下即可使用。这样使用的好处是,无需安装,简便;缺点是,当模板更新之后,你需要手动替换 \lstinline{cls} 文件。

\subsection{发行版安装与更新}

本模板测试环境为 
\begin{enumerate}
  \item Win10 + \TeX{} Live 2022;
  \item Ubuntu 20.04 + \TeX{} Live 2022;
  \item macOS Monterey + Mac\TeX{} 2022。
\end{enumerate}

\TeX Live/Mac\TeX{} 的安装请参考啸行的\href{https://github.com/OsbertWang/install-latex-guide-zh-cn/releases/}{一份简短的关于安装 \LaTeX{} 安装的介绍}。

安装 \TeX{} Live 之后,安装后建议升级全部宏包,升级方法:使用 cmd 或 terminal 运行 \lstinline{tlmgr update --all},如果 tlmgr 需要更新,请使用 cmd 运行 \lstinline{tlmgr update --self},如果更新过程中出现了中断,请改用 \lstinline{tlmgr update --self --all --reinstall-forcibly-removed} 更新,也即

\begin{lstlisting}
tlmgr update --self 
tlmgr update --all
tlmgr update --self --all --reinstall-forcibly-removed
\end{lstlisting}

更多的内容请参考 \href{https://tex.stackexchange.com/questions/55437/how-do-i-update-my-tex-distribution}{How do I update my \TeX{} distribution?}

\subsection{其他发行版本}

由于宏包版本问题,本模板不支持 C\TeX{} 套装,请务必安装 TeX Live/Mac\TeX{}。更多关于 \TeX{} Live 的安装使用以及 C\TeX{} 与 \TeX{} Live 的兼容、系统路径问题,请参考官方文档以及啸行的\href{https://github.com/OsbertWang/install-latex-guide-zh-cn/releases/}{一份简短的关于安装 \LaTeX{} 安装的介绍}。


\section{关于提交}

出于某些因素的考虑,Elegant\LaTeX{} 项目自 2019 年 5 月 20 日开始,\textbf{不再接受任何非作者预约性质的提交}(pull request)!如果你想改进模板,你可以给我们提交 issues,或者可以在遵循协议(LPPL-1.3c)的情况下,克隆到自己仓库下进行修改。

\chapter{版本更新历史}

根据用户的反馈,我们不断修正和完善模板。由于 3.00 之前版本与现在版本差异非常大,在此不列出 3.00 之前的更新内容。


\datechange{2025/01/06}{ 新建项目文件夹 }

\begin{change}
  \item 基于 ElegantBook 模板开始进行初步的适应性修改。
\end{change}

\nocite{*}
\printbibliography[heading=bibintoc, title=\ebibname]
\appendix

\chapter{基本数学工具}


本附录包括了计量经济学中用到的一些基本数学,我们扼要论述了求和算子的各种性质,研究了线性和某些非线性方程的性质,并复习了比例和百分数。我们还介绍了一些在应用计量经济学中常见的特殊函数,包括二次函数和自然对数,前 4 节只要求基本的代数技巧,第 5 节则对微分学进行了简要回顾;虽然要理解本书的大部分内容,微积分并非必需,但在一些章末附录和第 3 篇某些高深专题中,我们还是用到了微积分。

\section{求和算子与描述统计量}

\textbf{求和算子} 是用以表达多个数求和运算的一个缩略符号,它在统计学和计量经济学分析中扮演着重要作用。如果 $\{x_i: i=1, 2, \ldots, n\}$ 表示 $n$ 个数的一个序列,那么我们就把这 $n$ 个数的和写为:

\begin{equation}
\sum_{i=1}^n x_i \equiv x_1 + x_2 +\cdots + x_n
\end{equation}



\end{document}

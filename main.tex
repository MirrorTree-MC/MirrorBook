\documentclass[lang=cn,newtx,10pt,scheme=chinese]{elegantbook}

\title{MirrorBook:棱镜树不完全生存指南}
\subtitle{棱镜之书 - 序章}

\author{Various Artists}
\institute{Mirror\LaTeX{} Project}
\date{January 7, 2025}
\version{0.1}
% \bioinfo{自定义}{信息}

\extrainfo{不要以为抹消过去,重新来过,即可发生什么改变。—— 比企谷八幡}

\setcounter{tocdepth}{3}

\logo{logo.png}
\cover{lyc_church.jpeg}

% 本文档命令
\usepackage{array}
\newcommand{\ccr}[1]{\makecell{{\color{#1}\rule{1cm}{1cm}}}}

% 修改标题页的橙色带
\definecolor{customcolor}{RGB}{32,178,170}
\colorlet{coverlinecolor}{customcolor}
\usepackage{cprotect}

\addbibresource[location=local]{reference.bib} % 参考文献,不要删除

\begin{document}

\maketitle
\frontmatter

\tableofcontents

\mainmatter

\chapter{棱镜树简介}

一般来说,你只需阅览此章节便可开始在服务器当中的快乐游玩,其中将囊括此服务器中添加的主要衍生内容、重要的公共建筑和常用指令集。简而言之,棱镜树(MirrorTree)是一个拥有顾名思义的世界观,并以此为故事主线进行游玩体验设计的 Minecraft 服务器。你可以在腐竹的个人博客上找到详细的世界观,目前包含 \href{https://vichain.cn/tale/30}{《棱镜树·序》},\href{https://vichain.cn/tale/41}{《棱镜树·灵》},\href{https://vichain.cn/tale/54}{《棱镜树·冬》} 三篇。同时,我们的联系方式也列在此处,欢迎关注。

\begin{itemize}
    \item 玩家 QQ 群:876535736(多水群)
    \item GitHub 地址:\href{https://github.com/MirrorTree-MC}{https://github.com/MirrorTree-MC}
    \item 民营 Wiki:\href{https://wiki.mirror.bearcabbage.top/}{https://wiki.mirror.bearcabbage.top/}
    \item 官方 Wiki:\href{https://wiki-mirror.bearcabbage.top/}{https://wiki-mirror.bearcabbage.top/}
\end{itemize}

\section{主要衍生内容}

\section{重要公共建筑}

\section{常用指令集合}

\chapter{生存技巧(特性)}

或许你正在诅咒胡萝卜——欢迎来到琪露诺的完美生存教室!

\chapter{世界观设定}

既然你已经耐心地看到了这里,不妨来更加深入地了解一下棱镜树世界观背后的更多设定。

\nocite{*}
% \printbibliography[heading=bibintoc, title=\ebibname]
\appendix

\chapter{版本更新历史}

你可以在下方找到自\textbf{新建文件夹}以来的版本更新记录,亦即摸鱼记录。

\datechange{2025/01/07}{新建项目文件夹}

\begin{change}
  \item 基于 ElegantBook 模板开始进行初步的适应性修改。
  \item 添加了基本的结构框架,等待 Mods 实装完成。
\end{change}

\chapter{玩家简介}

棱镜树并不大,所以列出主要活动玩家并非是一件困难的事情。了解更多细节也有助于和他们进行\textit{友好交流}(bushi)。

\end{document}
